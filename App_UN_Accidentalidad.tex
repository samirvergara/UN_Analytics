\documentclass[]{article}
\usepackage{lmodern}
\usepackage{amssymb,amsmath}
\usepackage{ifxetex,ifluatex}
\usepackage{fixltx2e} % provides \textsubscript
\ifnum 0\ifxetex 1\fi\ifluatex 1\fi=0 % if pdftex
  \usepackage[T1]{fontenc}
  \usepackage[utf8]{inputenc}
\else % if luatex or xelatex
  \ifxetex
    \usepackage{mathspec}
  \else
    \usepackage{fontspec}
  \fi
  \defaultfontfeatures{Ligatures=TeX,Scale=MatchLowercase}
\fi
% use upquote if available, for straight quotes in verbatim environments
\IfFileExists{upquote.sty}{\usepackage{upquote}}{}
% use microtype if available
\IfFileExists{microtype.sty}{%
\usepackage{microtype}
\UseMicrotypeSet[protrusion]{basicmath} % disable protrusion for tt fonts
}{}
\usepackage[margin=1in]{geometry}
\usepackage{hyperref}
\hypersetup{unicode=true,
            pdftitle={Informe Técnico de Accidentalidad},
            pdfauthor={Universidad Nacional de Colombia - Especialización en Analítica},
            pdfborder={0 0 0},
            breaklinks=true}
\urlstyle{same}  % don't use monospace font for urls
\usepackage{color}
\usepackage{fancyvrb}
\newcommand{\VerbBar}{|}
\newcommand{\VERB}{\Verb[commandchars=\\\{\}]}
\DefineVerbatimEnvironment{Highlighting}{Verbatim}{commandchars=\\\{\}}
% Add ',fontsize=\small' for more characters per line
\usepackage{framed}
\definecolor{shadecolor}{RGB}{248,248,248}
\newenvironment{Shaded}{\begin{snugshade}}{\end{snugshade}}
\newcommand{\AlertTok}[1]{\textcolor[rgb]{0.94,0.16,0.16}{#1}}
\newcommand{\AnnotationTok}[1]{\textcolor[rgb]{0.56,0.35,0.01}{\textbf{\textit{#1}}}}
\newcommand{\AttributeTok}[1]{\textcolor[rgb]{0.77,0.63,0.00}{#1}}
\newcommand{\BaseNTok}[1]{\textcolor[rgb]{0.00,0.00,0.81}{#1}}
\newcommand{\BuiltInTok}[1]{#1}
\newcommand{\CharTok}[1]{\textcolor[rgb]{0.31,0.60,0.02}{#1}}
\newcommand{\CommentTok}[1]{\textcolor[rgb]{0.56,0.35,0.01}{\textit{#1}}}
\newcommand{\CommentVarTok}[1]{\textcolor[rgb]{0.56,0.35,0.01}{\textbf{\textit{#1}}}}
\newcommand{\ConstantTok}[1]{\textcolor[rgb]{0.00,0.00,0.00}{#1}}
\newcommand{\ControlFlowTok}[1]{\textcolor[rgb]{0.13,0.29,0.53}{\textbf{#1}}}
\newcommand{\DataTypeTok}[1]{\textcolor[rgb]{0.13,0.29,0.53}{#1}}
\newcommand{\DecValTok}[1]{\textcolor[rgb]{0.00,0.00,0.81}{#1}}
\newcommand{\DocumentationTok}[1]{\textcolor[rgb]{0.56,0.35,0.01}{\textbf{\textit{#1}}}}
\newcommand{\ErrorTok}[1]{\textcolor[rgb]{0.64,0.00,0.00}{\textbf{#1}}}
\newcommand{\ExtensionTok}[1]{#1}
\newcommand{\FloatTok}[1]{\textcolor[rgb]{0.00,0.00,0.81}{#1}}
\newcommand{\FunctionTok}[1]{\textcolor[rgb]{0.00,0.00,0.00}{#1}}
\newcommand{\ImportTok}[1]{#1}
\newcommand{\InformationTok}[1]{\textcolor[rgb]{0.56,0.35,0.01}{\textbf{\textit{#1}}}}
\newcommand{\KeywordTok}[1]{\textcolor[rgb]{0.13,0.29,0.53}{\textbf{#1}}}
\newcommand{\NormalTok}[1]{#1}
\newcommand{\OperatorTok}[1]{\textcolor[rgb]{0.81,0.36,0.00}{\textbf{#1}}}
\newcommand{\OtherTok}[1]{\textcolor[rgb]{0.56,0.35,0.01}{#1}}
\newcommand{\PreprocessorTok}[1]{\textcolor[rgb]{0.56,0.35,0.01}{\textit{#1}}}
\newcommand{\RegionMarkerTok}[1]{#1}
\newcommand{\SpecialCharTok}[1]{\textcolor[rgb]{0.00,0.00,0.00}{#1}}
\newcommand{\SpecialStringTok}[1]{\textcolor[rgb]{0.31,0.60,0.02}{#1}}
\newcommand{\StringTok}[1]{\textcolor[rgb]{0.31,0.60,0.02}{#1}}
\newcommand{\VariableTok}[1]{\textcolor[rgb]{0.00,0.00,0.00}{#1}}
\newcommand{\VerbatimStringTok}[1]{\textcolor[rgb]{0.31,0.60,0.02}{#1}}
\newcommand{\WarningTok}[1]{\textcolor[rgb]{0.56,0.35,0.01}{\textbf{\textit{#1}}}}
\usepackage{graphicx,grffile}
\makeatletter
\def\maxwidth{\ifdim\Gin@nat@width>\linewidth\linewidth\else\Gin@nat@width\fi}
\def\maxheight{\ifdim\Gin@nat@height>\textheight\textheight\else\Gin@nat@height\fi}
\makeatother
% Scale images if necessary, so that they will not overflow the page
% margins by default, and it is still possible to overwrite the defaults
% using explicit options in \includegraphics[width, height, ...]{}
\setkeys{Gin}{width=\maxwidth,height=\maxheight,keepaspectratio}
\IfFileExists{parskip.sty}{%
\usepackage{parskip}
}{% else
\setlength{\parindent}{0pt}
\setlength{\parskip}{6pt plus 2pt minus 1pt}
}
\setlength{\emergencystretch}{3em}  % prevent overfull lines
\providecommand{\tightlist}{%
  \setlength{\itemsep}{0pt}\setlength{\parskip}{0pt}}
\setcounter{secnumdepth}{0}
% Redefines (sub)paragraphs to behave more like sections
\ifx\paragraph\undefined\else
\let\oldparagraph\paragraph
\renewcommand{\paragraph}[1]{\oldparagraph{#1}\mbox{}}
\fi
\ifx\subparagraph\undefined\else
\let\oldsubparagraph\subparagraph
\renewcommand{\subparagraph}[1]{\oldsubparagraph{#1}\mbox{}}
\fi

%%% Use protect on footnotes to avoid problems with footnotes in titles
\let\rmarkdownfootnote\footnote%
\def\footnote{\protect\rmarkdownfootnote}

%%% Change title format to be more compact
\usepackage{titling}

% Create subtitle command for use in maketitle
\providecommand{\subtitle}[1]{
  \posttitle{
    \begin{center}\large#1\end{center}
    }
}

\setlength{\droptitle}{-2em}

  \title{Informe Técnico de Accidentalidad}
    \pretitle{\vspace{\droptitle}\centering\huge}
  \posttitle{\par}
    \author{Universidad Nacional de Colombia - Especialización en Analítica}
    \preauthor{\centering\large\emph}
  \postauthor{\par}
      \predate{\centering\large\emph}
  \postdate{\par}
    \date{11/10/2019}

\usepackage{booktabs}
\usepackage{longtable}
\usepackage{array}
\usepackage{multirow}
\usepackage{wrapfig}
\usepackage{float}
\usepackage{colortbl}
\usepackage{pdflscape}
\usepackage{tabu}
\usepackage{threeparttable}
\usepackage{threeparttablex}
\usepackage[normalem]{ulem}
\usepackage{makecell}
\usepackage{xcolor}

\begin{document}
\maketitle

Portilla, Estefano Restrepo, Juan Esteban Sterling, Jhordy Vergara,
Samir

\hypertarget{introduccion}{%
\subsubsection{1. Introducción}\label{introduccion}}

El presente informe contiene la descripcion de la accidentalidad en la
ciudad de Medellín, contemplando registros de accidentes desde el año
2014 hasta mediados de 2019, donde se presenta información relevante
como ubicación exacta de cada accidente o tipo de accidente, siendo de
gran utilidad al momento de hacer análisis y tomar decisiones referentes
a los accidentes en la ciudad de Medellín, los cuales con acciones
acertadas pueden disminuir partiendo de la información suministrada.

La ciudad de Medellín, al albergar una población que puede considerarse
densa para su extensión, requiere de diversas estrategias para mitigar
las diferentes problematicas que afronta cada día, entre ellas el
transito vehicular. Teniendo presente que el parque automotor para la
ciudad crece cada vez a mayor medida y la adecuación de vias no es
suficientemente efectiva, conlleva a generar problemas como lo son los
accidentes vehiculares, los cuales deben ser tenidos en cuenta al
momento de crear políticas en pro de generar bienestar social, dando a
los ciudadanos información de gran interés para transitar en los
diferentes sectores de la ciudad.

La información de accidentalidad que contiene el actual informe, no solo
permite dar a conocer el comportamiento de los accidentes en la ciudad
de Medellin en los ultimos años, esta además brinda bases sobre las
cuales se pueden generar nuevas estrategias para mitigar este tipo de
riesgos, impactando en lo que concierne a generar bienestar en el
transito vehicular para la ciudad de Medellín, y en cierta medida
mejorar la movilidad al crear soluciones basadas en datos reales

\hypertarget{justificacion}{%
\subsubsection{2. Justificación}\label{justificacion}}

Para la ciudad de Medellín, se hace necesario un análisis de
accidentalidad dada la cantidad de vehículos circulando en la ciudad y
el impacto en la movilidad y por ende en la calidad de vida de cada
habitante. Actualmente en la ciudad se encuentran en circulación
alrededor de 280 mil nuevos vehículos, los cuales fueron matriculados
durante el ultimo año, donde aproximadamente el 60\% de estos son motos.

Para nadie es ajeno que la probabilidad de accidente de una moto es
mucho mayor a la de un carro, y dado el alto porcentaje de estas en la
ciudad de Medellin no es extraño que se presenten diversos accidentes
cada día donde se encuentran involucrados este tipo de vehiculos,
teniendo presente que estos accidentes deben ser mitigados en la medida
de lo posible, ya que no solo las motos generan accidentes, por lo cual
se hace necesario en primera instancia determinar la ocurrencia de estos
y de este modo obtener un panorama general de su comportamiento en la
ciudad, generando así una primera visualización y un punto de partida
para tomar decisiones respecto a como mitigar la accidentalidad.

La información presentada en el este informe, busca servir como
herramienta para aquellos que desean conocer el comportamiento de la
accidentalidad en la ciudad de Medellin, ya sea para contribuir a su
disminución o como fuente de información para tomar alguna decisión
respecto a su propia movilidad. La información suministrada es flexible
en cuanto a su usabilidad e intuitiva para su interpretación, sin dejar
a un lado su precisión, la cual genera confiabilidad para aquellos que
deseen disponer de ella y crear estrategias de acción basados en los
presentes resultados.

\hypertarget{objetivos}{%
\subsubsection{3. Objetivos}\label{objetivos}}

\begin{itemize}
\item
  Presentar de manera clara y precisa la información de accidentalidad
  de la ciudad de Medellín, facilitando su interpretación mediante
  sencillos y prácticos análisis
\item
  Elaborar una aplicación web dinámica de la accidentalidad en Medellín
  usando herramientas openSource de visualización
\item
  Generar conciencia en los ciudadanos de la ciudad de Medellín acerca
  de la responsabilidad al transitar en cualquier tipo de vehículo,
  partiendo de la información de accidentalidad presentada
\end{itemize}

\hypertarget{datos}{%
\subsubsection{4. Datos}\label{datos}}

\hypertarget{datos-abiertos-de-movilidad---alcaldia-de-medellin}{%
\subparagraph{Datos Abiertos de Movilidad - Alcaldía de
Medellín}\label{datos-abiertos-de-movilidad---alcaldia-de-medellin}}

Los datos de accidentalidad utilizados en este informe corresponden a
registros entre 2014 y 2018, los cuales fueron descargados del sitio:
\href{https://geomedellin-m-medellin.opendata.arcgis.com/search?tags=movilidad}{Alcaldía
de Medellín - OpenData: Movilidad}

\hypertarget{volumetria-numero-de-accidentes-por-ano-y-mes}{%
\paragraph{4.1 Volumetría (Número de accidentes por año y
mes)}\label{volumetria-numero-de-accidentes-por-ano-y-mes}}

En la siguiente tabla se puede apreciar la cantidad de accidentes por
año y mes en la ciudad de Medellín:

\begingroup\fontsize{12}{14}\selectfont

\begin{tabular}{l|r|r|r|r|r|r|r|r|r|r|r|r}
\hline
  & 1 & 2 & 3 & 4 & 5 & 6 & 7 & 8 & 9 & 10 & 11 & 12\\
\hline
2014 & 2991 & 3258 & 3855 & 3425 & 3686 & 3404 & 3620 & 3572 & 3533 & 3429 & 3318 & 3503\\
\hline
2015 & 2950 & 3291 & 3635 & 3418 & 3679 & 3254 & 3708 & 3729 & 3789 & 3666 & 3376 & 3585\\
\hline
2016 & 3028 & 3513 & 3588 & 3620 & 3579 & 3360 & 3616 & 3890 & 3734 & 3844 & 3558 & 3511\\
\hline
2017 & 3136 & 3492 & 3566 & 3351 & 3742 & 3459 & 3606 & 3906 & 3767 & 3594 & 3435 & 3509\\
\hline
2018 & 2959 & 3251 & 3334 & 3382 & 3529 & 3291 & 3334 & 3598 & 3462 & 3535 & 3264 & 3409\\
\hline
2019 & 2923 & 3136 & 3154 & 3156 & 3480 & 3264 & 0 & 0 & 0 & 0 & 0 & 0\\
\hline
\multicolumn{13}{l}{\textit{--}}\\
\multicolumn{13}{l}{Archivos fuente del análisis...}\\
\multicolumn{13}{l}{\textsuperscript{1} Accidentalidad\_georreferenciada\_2014.csv; }\\
\multicolumn{13}{l}{\textsuperscript{2} Accidentalidad\_georreferenciada\_2015.csv; }\\
\multicolumn{13}{l}{\textsuperscript{3} Accidentalidad\_georreferenciada\_2016.csv; }\\
\multicolumn{13}{l}{\textsuperscript{4} Accidentalidad\_georreferenciada\_2017.csv; }\\
\multicolumn{13}{l}{\textsuperscript{5} Accidentalidad\_georreferenciada\_2018.csv; }\\
\multicolumn{13}{l}{\textsuperscript{6} Accidentalidad\_georreferenciada\_2019.csv; }\\
\end{tabular}
\endgroup{}

\hypertarget{enriquecimiento-de-los-datos}{%
\paragraph{4.2 Enriquecimiento de los
datos}\label{enriquecimiento-de-los-datos}}

Hemos enriquecido los datos con algunas variables que se pueden
determinar desde la fuente de datos original, a continuación se listan:

\hypertarget{variables-latitud-y-longitud}{%
\subparagraph{4.2.1 Variables: LATITUD y
LONGITUD}\label{variables-latitud-y-longitud}}

Variables usadas para georeferenciación, ubicamos la librería proj4 que
nos permite convertir las coordenadas geodésicas de la fuente
(``X'',``Y'') al Sistema de coordenadas universal transversal de
Mercator (UTM): (``LATITUD'', ``LONGITUD'')

\begin{Shaded}
\begin{Highlighting}[]
\CommentTok{#library(proj4)}

\NormalTok{lib <-}\StringTok{ "+proj=tmerc +lat_0=4.596200416666666 +lon_0=-74.07750791666666 +k=1 +x_0=1000000 +y_0=1000000 +ellps=WGS84 +towgs84=0,0,0,0,0,0,0 +units=m +no_defs"}
\end{Highlighting}
\end{Shaded}

\hypertarget{limpieza-de-datos}{%
\paragraph{4.3 Limpieza de datos}\label{limpieza-de-datos}}

En algunas de las variables de análisis hemos detectado que la calidad
de los datos no es la deseada, a continuación se listan las
transformaciones realizadas a los datos por cada una de las variables:

\hypertarget{variables-gravedad-y-dia-semana}{%
\subparagraph{4.3.1 Variables: GRAVEDAD y DÍA
SEMANA}\label{variables-gravedad-y-dia-semana}}

Se mejora el formato de los datos, se pasa de MAYÚSCULAS a Mayúscula
inicial:

\begingroup\fontsize{12}{14}\selectfont

\begin{tabular}{l}
\hline
GRAVEDAD\\
\hline
Solo daños\\
\hline
Muerto\\
\hline
Herido\\
\hline
\end{tabular}
\endgroup{}

\hypertarget{variable-comuna}{%
\subparagraph{4.3.2 Variable: COMUNA}\label{variable-comuna}}

Aquí construimos una tabla de homologación, con los 85 valores
disponibles en esta variable. Se estandariza y se le asigna a cada valor
una de las 16 comunas o 5 corregimientos de Medellín. Fuente Wikipedia:
\href{https://es.wikipedia.org/wiki/Anexo:Barrios_de_Medell\%C3\%Adn}{Anexo:Barrios
de Medellín}

\begingroup\fontsize{12}{14}\selectfont

\begin{tabular}{l|r|r|l|l}
\hline
COMUNA\_ORI & FREQ & COMUNA\_ID & COMUNA\_STD & TIPO\\
\hline
 & 489 & 0 & No Data & Comuna\\
\hline
Aranjuez & 12425 & 4 & Aranjuez & Comuna\\
\hline
AU & 17 & 0 & No Data & Comuna\\
\hline
Belén & 13933 & 16 & Belén & Comuna\\
\hline
Buenos Aires & 7791 & 9 & Buenos Aires & Comuna\\
\hline
Castilla & 21208 & 5 & Castilla & Comuna\\
\hline
\end{tabular}
\endgroup{}

\hypertarget{variable-clase}{%
\subparagraph{4.3.3 Variable: CLASE}\label{variable-clase}}

Aquí lo que hicimos fue reemplazar los valores: ``Caída de Ocupante''
por ``Caída Ocupante'' y ``Choque y Atropello'' por ``Choque''
\begingroup\fontsize{12}{14}\selectfont

\begin{tabular}{l}
\hline
CLASE\\
\hline
Choque\\
\hline
Volcamiento\\
\hline
Otro\\
\hline
Caída Ocupante\\
\hline
Atropello\\
\hline
Incendio\\
\hline
\\
\hline
\end{tabular}
\endgroup{}

\hypertarget{variable-hora}{%
\subparagraph{4.3.4 Variable: HORA}\label{variable-hora}}

Vamos a convertir los 2 formatos hora (``08:45 AM'' y ``8:15:00 p.~m.'')
hallados en la fuente en formato ``24H'': Del ``00'' a ``23'' para poder
analizar la variable HORA

\hypertarget{variables-de-analisis}{%
\subparagraph{4.4 Variables de análisis}\label{variables-de-analisis}}

Finalmente las variables seleccionadas para el análisis de los datos en
la App en Shiny son:

\begingroup\fontsize{12}{14}\selectfont

\begin{tabular}{l|l|l|r|l|l|l|r|r|r|r|l}
\hline
  & RANGO\_HORA & DIA\_NOMBRE & PERIODO & CLASE & GRAVEDAD & COMUNA & MES & DIA & LATITUD & LONGITUD & FECHA\\
\hline
514 & 10 & Jueves & 2017 & Choque & Muerto & Buenos Aires & 11 & 9 & 6.2354989 & -75.543339 & 2017-11-09\\
\hline
515 & 11 & Viernes & 2017 & Atropello & Muerto & Castilla & 8 & 18 & 6.2836868 & -75.576608 & 2017-08-18\\
\hline
516 & 08 & Jueves & 2017 & Choque & Muerto & Altavista & 12 & 21 & 6.2251595 & -75.606824 & 2017-12-21\\
\hline
517 & 10 & Lunes & 2017 & Choque & Muerto & Santa Cruz & 12 & 25 & 6.2996625 & -75.557446 & 2017-12-25\\
\hline
518 & 10 & Lunes & 2017 & Choque & Muerto & Santa Cruz & 12 & 25 & 6.2996625 & -75.557446 & 2017-12-25\\
\hline
519 & 10 & Lunes & 2017 & Choque & Muerto & Santa Cruz & 12 & 25 & 6.2996625 & -75.557446 & 2017-12-25\\
\hline
\end{tabular}
\endgroup{}

\hypertarget{evidencias}{%
\subsubsection{5. Evidencias}\label{evidencias}}

\hypertarget{mapa-de-calor-de-accidentalidad-por-ano-y-mes}{%
\paragraph{5.1 Mapa de calor de accidentalidad por año y
mes}\label{mapa-de-calor-de-accidentalidad-por-ano-y-mes}}

A continuación vemos una representación gráfica por años en la que se
visualizan los meses donde más se focalizan los accidentes en Medellín:
Más accidentes: - Mayo: El 2do mes más lluvioso del año - Agosto:
Coincide con la Feria de las Flores - Mientras que Enero es el menos con
menos accidentes lo cual coincide con temporada de vacaciones y el mes
menos lluvioso del año

\includegraphics{App_UN_Accidentalidad_files/figure-latex/headmap-1.pdf}

\hypertarget{tendencia-por-horas-de-la-semana}{%
\paragraph{5.2 Tendencia por horas de la
semana}\label{tendencia-por-horas-de-la-semana}}

Se puede apreciar que la mayor cantidad de accidentes coincide con los
horarios cercanos a pico y placa: De las 06 a las 08 de la mañana y de
las 17 a las 18 horas en la tarde, luego sigue la franja del medio día
entre las 12 y las 14 horas (Hora de almuerzo).

\includegraphics{App_UN_Accidentalidad_files/figure-latex/unnamed-chunk-11-1.pdf}

\hypertarget{accidentalidad-por-mes}{%
\paragraph{5.3 Accidentalidad por mes}\label{accidentalidad-por-mes}}

A continuación un gráfico de barras apilado por años y meses:

\includegraphics{App_UN_Accidentalidad_files/figure-latex/unnamed-chunk-12-1.pdf}

\hypertarget{accidentalidad-por-comuna}{%
\paragraph{5.4 Accidentalidad por
comuna}\label{accidentalidad-por-comuna}}

A continuación un gráfico de barras comunas (Se usa la variable COMUNA
estandarizda):

\includegraphics{App_UN_Accidentalidad_files/figure-latex/unnamed-chunk-13-1.pdf}

\hypertarget{conclusiones}{%
\subsubsection{6. Conclusiones}\label{conclusiones}}

Con la información presentada sobre la accidentalidad en la ciudad de
Medellin, se logran identificar parametros importantes que sirven de
referencia al momento de tomar medidas de precaución frente a este tipo
de riesgos, ya que se muestran indicadores de accidentes en diferentes
modalidades de tiempo, localización y severidad, los cuales sirven como
punto de partida en la definicion de criterios para la articulación de
posibles soluciones asociadas a la accidentalidad

La presentación de la información se logra efectuar de forma clara y
dinámica, de manera que pueda ser interpretada por la mayoria de las
personas de la ciudad de Medellín, ya que son los principales
beneficiarios y en primera instancia quienes toman decisiones propias al
momento de salir a circular en vehículos dentro de la ciudad de Medellín

La información presentada sobre accidentalidad en la ciudad de Medellin,
genera un efecto considerable en cuanto a conocimiento general del
comportamiento de los accidentes en la ciudad, y a su vez sirve de
herramienta informativa para tomar precauciones al momento de transitar
en la ciudad e Medellín, ya sea con o sin un vehículo

\hypertarget{bibliografia}{%
\subsubsection{7. Bibliografía}\label{bibliografia}}

\begin{itemize}
\item
  El portal de datos de Medellín
  \href{https://geomedellin-m-medellin.opendata.arcgis.com/search?tags=movilidad}{Alcaldía
  de Medellín - OpenData: Movilidad}
\item
  Alcaldía de Medellín:
  \href{https://www.medellin.gov.co/movilidad/cifras-estudios/viewcategory/1872-parque-automotor}{Secretaría
  de movilidad de Medellín}
\item
  Periódico El Tiempo:
  \href{https://www.eltiempo.com/colombia/medellin/el-valle-de-aburra-supero-las-900-000-motos-matriculadas-343266}{El
  Tiempo}
\item
  Revista Dinero:
  \href{https://www.dinero.com/economia/articulo/cuantos-carros-y-motos-se-vendieron-en-enero-de-2019/266687}{Dinero}
\item
  Alianza interinstitucional
  \href{https://www.medellincomovamos.org/la-ciudad/}{Medellín cómo
  vamos}
\item
  Wikipedia
  \href{https://es.wikipedia.org/wiki/Anexo:Barrios_de_Medell\%C3\%ADn}{Barrios
  de Medellín}
\end{itemize}


\end{document}
